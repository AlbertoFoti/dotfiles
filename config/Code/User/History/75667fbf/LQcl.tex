\chapter{Nominal Trajectory}

The \textit{Descent Module} (DM), during the EDL phase, can be modeled as a
six dof (9dof) autonomous system.

It has three coordinates (spherical) describing its position relative to the
$\mathcal{R}_{m}$ reference frame

        $$
            \textbf{r}_{CoM}\rangle_{\mathcal{R}_{P}} = [\: r,\:\Lambda,\:\lambda \:]
        $$

\noindent Where $r$ is the radial position w.r.t. the center of the planet, $\Lambda$ and $\lambda$ are 
the longitude and the latitude respectively.

Three coordinates (spherical) describing the velocity vector w.r.t the $\mathcal{R}_{NED}$ reference frame, univocally
determined by the DM position.

        $$
            \textbf{v}_{CoM}\rangle_{\mathcal{R}_{NED}} = [\: \nu,\:\gamma,\:\chi \:]
        $$

\noindent Where $\nu$ is the velocity modulus (VEL), while $\gamma$ and $\chi$ are the flight
path angle (FPA) and the heading (HEAD) respectively.
These two can be combined in a six dof description of the capsule state.

        $$
            \textbf{x}_{CoM}\rangle = [\textbf{r}_{CoM},\: \textbf{v}_{CoM}] = [\: r,\:\Lambda,\:\lambda, \: \nu,\:\gamma,\:\chi \:]
        $$

Additionally, a 3D vector is used to describe the DM body attitude w.r.t. to the relative wind.

        $$
            \textbf{x}_{Att} = [\: \mu,\: \alpha,\: \eta]
        $$

\noindent Where $\mu$ is the bank angle, while $\alpha$ and $\eta$ are the 
angle of attack (AoA) and the aerodynamic roll.

The roll, pitch and yaw rates around the body $\mathcal{R}_{b}$ reference frame are the DM angular velocity

        $$
            \boldsymbol{\omega}| = [\: \mu,\: \alpha,\: \eta]
        $$


\newpage

\section{Nominal Trajectory Description}

from the DM known state (\textbf{x})