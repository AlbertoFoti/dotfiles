\chapter{Nominal Trajectory}

The \textit{Descent Module} (DM), during the EDL phase, can be modeled as a
six dof (9dof) autonomous system.

It has three coordinates (spherical) describing its position relative to the
$\mathcal{R}_{m}$ reference frame

        $$
            \textbf{r}_{CoM}\rangle_{\mathcal{R}_{P}} = [\: r,\:\Lambda,\:\lambda \:]
        $$

\noindent Where $r$ is the radial position w.r.t. the center of the planet, $\Lambda$ and $\lambda$ are 
the longitude and the latitude respectively.

Three coordinates (spherical) describing the velocity vector w.r.t the $\mathcal{R}_{NED}$ reference frame, univocally
determined by the DM position.

        $$
            \textbf{v}_{CoM}\rangle_{\mathcal{R}_{NED}} = [\: \nu,\:\gamma,\:\chi \:]
        $$

\noindent Where $\nu$ is the velocity modulus (VEL), while $\gamma$ and $\chi$ are the flight
path angle (FPA) and the heading (HEAD) respectively.
These two can be combined in a six dof description of the capsule state.

        $$
            \textbf{x}_{CoM}\rangle = [\textbf{r}_{CoM},\: \textbf{v}_{CoM}] = [\: r,\:\Lambda,\:\lambda, \: \nu,\:\gamma,\:\chi \:]
        $$

Additionally, a 3D vector is used to describe the DM body attitude w.r.t. to the relative wind.

        $$
            \textbf{x}_{Att} = [\: \mu,\: \alpha,\: \eta]
        $$

\noindent Where $\mu$ is the bank angle, while $\alpha$ and $\eta$ are the 
angle of attack (AoA) and the aerodynamic roll.

The roll, pitch and yaw rates around the body $\mathcal{R}_{b}$ reference frame are the DM angular velocity

        $$
            \boldsymbol{\omega}|_{\mathcal{R}_{b}} = [\: \omega_l,\: \omega_m,\: \omega_n]
        $$

\newpage

\section{Errors description and notations}

There are, in general, two types of errors: actual tracking errors and estimation
errors, whice can also be called dispersion and knowledge errors respectively.
The tracking error is defined as the difference between the actual state of the capsule and the state
at which it's supposed to be, while the estimation error is the difference between the estimated state
(through the use of sensors) and the actual one.

Actually there are several instances of the capsule state. The ones used in this document are:

\begin{enumerate}
    \item Actual Trajectory: Represented as $\textbf{x}_{CoM}$ and $\textbf{x}_{att}$, correspond
    to the actual value of the DM state, given, during the design phase by the simulation engine.
    \item Navigated Trajectory: Represented as $\hat{\textbf{x}}_{CoM}$ and $\hat{\textbf{x}}_{att}$,
    corresponds to the estimated capsule state obtained by the navigation algorithm by interpreting the
    information of all the available sensors.
    \item Mission Trajectory: Represented as $\textbf{x}_{CoM,Mis}$ and $\textbf{x}_{att,Mis}$
\end{enumerate}

\newpage

\section{Nominal Trajectory Description}

from the DM known state (\textbf{x})