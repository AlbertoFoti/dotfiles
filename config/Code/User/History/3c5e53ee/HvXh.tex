\chapter{Introduction}

    \section{Entry Phase of a Mission}

        An \textbf{\textit{Entry Phase}} of a mission encompasses all operations and
        events involving the capsule atmospheric entry into the target planet
        atmosphere up until touchdown completion.

        \textbf{Separation of the carrier point (SCP)}

        \textbf{Entry interface point (EIP)}

        \textbf{Peak Heat Flux Point (HFP) and Peak Load Factor Point (LFP)}

        \textbf{Parachute deployment point (PDP)}

        \textbf{Propulsive Landing Initiation (PLP)}

        \textbf{Landing Site (LS)}

        The intervals between these events are called \textbf{\textit{Phases}}.

        \textbf{Coasting (CST): From SCP to EIP}

        \textbf{Atmospheric Entry Phase (ENT): From EIP to PDP}

        \textbf{Parachute Phase (PAR): From PDP to PLP}

        \textbf{Terminal Descent and Landing Phase (TDL): From PLP to LS}


    \section{Guided Entry (GE)}

        \textbf{\textit{Guided Entry}} (GE) techniques can be employed in the Atmospheric Entry Phase (ENT)
        between the Entry Interface Point (EIP) and the Parachute Deployment Point (PDP).

