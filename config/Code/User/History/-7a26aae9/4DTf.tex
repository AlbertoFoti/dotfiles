\newpage
\chapter{Notation and Conventions}

   \begin{enumerate}
      \item An \textbf{\textit{Inertial Reference Frame}} is defined given by the attitude and orbit of the target planet
         and the stars, at a fixed time.
         The origin is assumed fixed at the center of the planet.
         $\mathcal{R}_{I}=\{ \mathcal{O}_{I},\: \textbf{i}_{I},\:\textbf{j}_I,\: \textbf{k}_I \}$

      \item A \textbf{\textit{Target Planet reference Frame}} is defined with 
         origin as $\mathcal{R}_{I}$ and rotating with respect to it with an angular
         velocity $\omega_{P}\cdot \mathcal{O}_{P}$.

      \item 
   
   \end{enumerate}


\newpage
\subsubsection{NED Reference Frame}

   The common reference frame is the LVLH (Local Vertical Local Horizontal), in the ENU (East, North, Up) variant.
   The naming convention for the axis follow the standard for the general LVLH, therefore the following equivalences are valid

   $$
      East\: (E) \quad\to\quad x
   $$

   $$
      North\: (N) \quad\to\quad y
   $$

   $$
      Up (U)\: \quad\to\quad z
   $$

Further details on other reference frames used can be found in Appendix A.