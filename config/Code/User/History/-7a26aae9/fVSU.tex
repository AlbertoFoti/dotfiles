\newpage
\chapter{Notation and Conventions}

\subsubsection{Reference Frames}

   \begin{enumerate}
      \item An \textbf{\textit{Inertial Reference Frame}} is defined given by the attitude and orbit of the target planet
         and the stars, at a fixed time.
         The origin is assumed fixed at the center of the planet.

         $$\mathcal{R}_{I}=\{ \mathcal{O}_{I},\: \textbf{i}_{I},\:\textbf{j}_I,\: \textbf{k}_I \}$$

      \item A \textbf{\textit{Target Planet reference Frame}} is defined with 
         origin as $\mathcal{R}_{I}$ and rotating with respect to it with an angular
         velocity $\omega_{P}\cdot \mathcal{O}_{P}$.
         $\textbf{i}_{P}$ points to the prime meridian, $\textbf{k}_{P}$ points to the
         north pole (the rotation axis) and $\textbf{j}_{P}$ completes according
         to the right-hand-rule.

         $$\mathcal{R}_{P}=\{ \mathcal{O}_{P},\: \textbf{i}_{P},\:\textbf{j}_P,\: \textbf{k}_P \}$$

         The capsule position spherical coordinates on $\mathcal{R}_{I}$ are the radial position
         ($r$), longitude ($\Lambda$) and latitude ($\lambda$) of the capsule.
      
         $$
            \textbf{x}_{CoM} = [\: r,\:\Lambda,\:\lambda \:]
         $$

      \item A \textbf{\textit{North, East, Down (NED) Reference Frame}} is defined, with origin $\mathcal{O}_{NED}$
         corresponding to the capsule Center of Mass (CoM) position, with $\textbf{i}_{NED}$
         pointing tangent to the curve going to the North Pole or the target planet (extended at
         the altitude of the capsule), $\textbf{j}_{NED}$ pointing East and $\textbf{k}_{NED}$
         pointing directly to the center of the target planet (downwards).

         $$\mathcal{R}_{NED}=\{ \mathcal{O}_{NED},\: \textbf{i}_{NED},\:\textbf{j}_{NED},\: \textbf{k}_{NED} \}$$

         The co-rotating velocity ($\textbf{v}_{p}$), in spherical coordinates on $\mathcal{R}_{NED}$, is the velocity module ($v_{m}$), 
         the flight path angle (FPA, $\gamma$), and heading ($\chi$) of the capsule.
      
         $$
            \textbf{v}_{CoM} = [\: v_{p},\:\gamma,\:\chi \:]
         $$

      \item The \textbf{\textit{Local Vertical (LV) Reference Frame}} can be obtained from
         the $\mathcal{R}_{NED}$ by a rotation of angle $\chi$ around $\textbf{k}_{NED}$.

      \item The \textbf{Velocity (V) Reference Frame} is defined by rotating the LV reference frame
         of angle $\gamma$ around the $\textbf{j}_{LV}$ vector. As a result, the $\textbf{i}_{v}$ vector points
         in the direction of the velocity $\textbf{v}_{p}$

         $$\mathcal{R}_{v}=\{ \mathcal{O}_{v},\: \textbf{i}_{v},\:\textbf{j}_{v},\: \textbf{k}_{v} \}$$
   \end{enumerate}

   \vspace{0.5cm}

   Since the aerodynamic forces depend on the attitude of the capsule wrt its velocity (assuming no wind), it's
   useful to describe the attitude of $\mathcal{R}_{v}$ wrt $\mathcal{R}_{b}$ (body reference frame).

   The \textbf{Body (B) Reference Frame} can be defined starting from the $\mathcal{R}_{v}$ reference frame and
   applying the following rotations (1-2-1):

   \begin{enumerate}
      \item One rotation is needed to arrive from $\mathcal{R}_v$ to $\mathcal{R}_w$ about $\textbf{i}_v$ providing
         the \textbf{Aerodynamic bank angle $\mu$}.
      \item One rotation is needed to arrive from $\mathcal{R}_w$ to $\mathcal{R}_{as}$ about $\textbf{j}_w$ providing
         the \textbf{Total Angle of Attack (AoA, $\alpha$)}.
      \item One rotation is needed to arrive from $\mathcal{R}_{as}$ to $\mathcal{R}_{b}$ about $\textbf{i}_{as}$ providing
         the \textbf{Aerodynamic Roll ($\eta$)}.
   \end{enumerate}

   The intermediate reference frames \textbf{Wind (w) Reference Frame} and \textbf{Axis Symmentric (AS) Reference Frame} are called
   $\mathcal{R}_w$ and $\mathcal{R}_{as}$.

   Two additional reference frames can be defined:

   \begin{enumerate}
      \item \textbf{Aerodynamic Data Base}: $\mathcal{R}_{a}$ fixed to the capsule, is parallel to $\mathcal{R}_b$ and considers
         only a displacement of $\mathcal{O}_a$ wrt $\mathcal{O}_b$ = CoM . This reference frame is needed since the aerodynamic
         characteristics (coefficients) of the capsule are given wrt to $\mathcal{R}_a$ , not to $\mathcal{R}_b$ . In fact, it is possible that
         $\mathcal{O}_b$ moves wrt $\mathcal{O}_a$ during the entry phased.
      \item \textbf{Descent Module Reference Frame}: $\mathcal{R}_{DMRF}$ fixed to the capsule, is
      parallel to $\mathcal{R}_b$ . Its origin is placed at the back of the back shell of the capsule over the symmetry
      axis. This reference frame is used as the reference frame of the descent module to which all
      numerical coordinates should be referred to avoid ambiguity.
   \end{enumerate}

Further details on other reference frames used can be found in Appendix A.