\newpage
\chapter{Notation and Conventions}

   \begin{enumerate}
      \item An \textbf{\textit{Inertial Reference Frame}} is defined given by the attitude and orbit of the target planet
         and the stars, at a fixed time.
         The origin is assumed fixed at the center of the planet.

         $$\mathcal{R}_{I}=\{ \mathcal{O}_{I},\: \textbf{i}_{I},\:\textbf{j}_I,\: \textbf{k}_I \}$$

      \item A \textbf{\textit{Target Planet reference Frame}} is defined with 
         origin as $\mathcal{R}_{I}$ and rotating with respect to it with an angular
         velocity $\omega_{P}\cdot \mathcal{O}_{P}$.
         $\textbf{i}_{P}$ points to the prime meridian, $\textbf{k}_{P}$ points to the
         north pole (the rotation axis) and $\textbf{j}_{P}$ completes according
         to the right-hand-rule.

         $$\mathcal{R}_{P}=\{ \mathcal{O}_{P},\: \textbf{i}_{P},\:\textbf{j}_P,\: \textbf{k}_P \}$$

         The capsule position spherical coordinates on $\mathcal{R}_{I}$ are the radial position
         ($r$), longitude ($\Lambda$) and latitude ($\lambda$) of the capsule.
      
         $$
            \textbf{x}_{CoM} = [\: r,\:\Lambda,\:\lambda \:]
         $$

      \item A \textbf{\textit{North, East, Down (NED) Reference Frame}} is defined, with origin $\mathcal{O}_{NED}$
         corresponding to the capsule Center of Mass (CoM) position, with $\textbf{i}_{NED}$
         pointing tangent to the curve going to the North Pole or the target planet (extended at
         the altitude of the capsule), $\textbf{j}_{NED}$ pointing East and $\textbf{k}_{NED}$
         pointing directly to the center of the target planet (downwards).

         $$\mathcal{R}_{NED}=\{ \mathcal{O}_{NED},\: \textbf{i}_{NED},\:\textbf{j}_{NED},\: \textbf{k}_{NED} \}$$

         The co-rotating velocity ($\textbf{v}_{m}$), in spherical coordinates on $\mathcal{R}_{NED}$, is the velocity module ($v_{m}$), 
         the flight path angle (FPA, $\gamma$), and heading ($\chi$) of the capsule.
      
         $$
            \textbf{v}_{CoM} = [\: v_{m},\:\gamma,\:\chi \:]
         $$

      \item The \textbf{\textit{Local Vertical (LV) Reference Frame}} can be obtained from
         the $\mathcal{R}_{NED}$ by a rotation of angle $\chi$ around $\textbf{k}_{NED}$.

      \item The \textbf{Velocity (V) Reference Frame} is defined by rotating the LV reference frame
         of angle $\gamma$ around the $\textbf{j}_{LV}$ vector.

         $$\mathcal{R}_{V}=\{ \mathcal{O}_{V},\: \textbf{i}_{V},\:\textbf{j}_{V},\: \textbf{k}_{NED} \}$$

   \end{enumerate}

   


\newpage
\subsubsection{NED Reference Frame}

   The common reference frame is the LVLH (Local Vertical Local Horizontal), in the ENU (East, North, Up) variant.
   The naming convention for the axis follow the standard for the general LVLH, therefore the following equivalences are valid

   $$
      East\: (E) \quad\to\quad x
   $$

   $$
      North\: (N) \quad\to\quad y
   $$

   $$
      Up (U)\: \quad\to\quad z
   $$

Further details on other reference frames used can be found in Appendix A.